
\documentclass[11pt]{article}
\usepackage{fullpage}
\usepackage{siunitx}
\usepackage{hyperref,graphicx,booktabs,dcolumn}
\usepackage{stata}
\usepackage[x11names]{xcolor}
\bibliographystyle{unsrt}
\usepackage{natbib}

\usepackage{chngcntr}
\counterwithin{figure}{section}
\counterwithin{table}{section}

\usepackage{multirow}
\usepackage{booktabs}

\newcommand{\specialcell}[2][c]{%
  \begin{tabular}[#1]{@{}c@{}}#2\end{tabular}}
\newcommand{\thedate}{\today}

\usepackage{pgfplotstable}

\begin{document}


\begin{titlepage}
    \begin{flushright}
        \Huge
        \textbf{International trends in the incidence of diabetes in young people}
\color{black}
\rule{16cm}{2mm} \\
\Large
\color{black}
\thedate \\
\color{blue}
https://github.com/jimb0w/YO \\
\color{black}
       \vfill
    \end{flushright}
        \Large

\noindent
Jedidiah Morton \\
\color{blue}
\href{mailto:Jedidiah.Morton@Monash.edu}{Jedidiah.Morton@monash.edu} \\ 
\color{black}
Research Fellow \\
\color{blue}
\color{black}
Monash University, Melbourne, Australia \\\
Baker Heart and Diabetes Institute, Melbourne, Australia \\
\\
\noindent
Lei Chen \\
Research Officer \\
Baker Heart and Diabetes Institute, Melbourne, Australia \\
\\
\noindent
Bendix Carstensen \\
Senior Statistician \\
Steno Diabetes Center Copenhagen, Gentofte, Denmark \\
Department of Biostatistics, University of Copenhagen \\
\\
\noindent
Dianna Magliano \\
Professor and Head of Diabetes and Population Health \\
Baker Heart and Diabetes Institute, Melbourne, Australia \\

\end{titlepage}

\pagebreak
\tableofcontents


\pagebreak
\section{Preface}

The methods used in this analyses are drawn heavily/almost entirely from Bendix Carstensen 
(see \cite{MaglianoLDE2021,CarstensenSTATMED2007}). \\
To generate this document, the Stata package texdoc \cite{Jann2016Stata} was used, which is 
available from: \color{blue} \url{http://repec.sowi.unibe.ch/stata/texdoc/} \color{black} (accessed 14 November 2022). The 
final Stata do file and this pdf are available at: \color{blue} \url{https://github.com/jimb0w/YO} \color{black}.
The ordinal colour schemes used are \emph{inferno} and \emph{viridis} from the
\emph{viridis} package \cite{GarnierR2021}.

\pagebreak
\section{Crude rates}

We start by examining crude incidence rates for each country.
We will generate a table showing the overall counts for each country, then plots of the crude incidence
of each type of diabetes by sex and year. 
Also, because the diabetes type definitions require two years of non-insulin use to
be effective, we will drop all data from 2021 or later. 

\color{Blue4}
\begin{stlog}\input{log/1.log.tex}\end{stlog}
\color{black}

\begin{table}[h!]
  \begin{center}
    \caption{Incident diabetes cases and person-years of follow-up in people without diabetes for people aged 15-39, by country and sex.}
    \label{T1}
     \fontsize{7pt}{9pt}\selectfont\pgfplotstabletypeset[
      multicolumn names,
      col sep=colon,
      header=false,
      string type,
	  display columns/0/.style={column name=Country,
		assign cell content/.code={
\pgfkeyssetvalue{/pgfplots/table/@cell content}
{\multirow{2}{*}{##1}}}},
	  display columns/1/.style={column name=Period,
		assign cell content/.code={
\pgfkeyssetvalue{/pgfplots/table/@cell content}
{\multirow{2}{*}{##1}}}},
      display columns/2/.style={column name=Sex, column type={l}, text indicator="},
      display columns/3/.style={column name=Type 1 diabetes, column type={r}},
      display columns/4/.style={column name=Type 2 diabetes, column type={r}},
      display columns/5/.style={column name=Uncertain diabetes type, column type={r}},
      display columns/6/.style={column name=\specialcell{Person-years in people \\ without diabetes}, column type={r}},
      every head row/.style={
        before row={\toprule
					},
        after row={\midrule}
            },
        every nth row={2}{before row=\midrule},
        every last row/.style={after row=\bottomrule},
    ]{T1.csv}
  \end{center}
\end{table}

\color{Blue4}
\begin{figure}
    \centering
    \includegraphics[width=0.8\textwidth]{log/2.pdf}
    \caption{Crude incidence of diabetes in Australia among people aged 15-39 years, by diabetes type}
    \label{Australiacrude}
\end{figure}
\begin{figure}
    \centering
    \includegraphics[width=0.8\textwidth]{log/2_1.pdf}
    \caption{Crude incidence of diabetes in Catalonia, Spain among people aged 15-39 years, by diabetes type}
    \label{Catalonia, Spaincrude}
\end{figure}
\begin{figure}
    \centering
    \includegraphics[width=0.8\textwidth]{log/2_2.pdf}
    \caption{Crude incidence of diabetes in Denmark among people aged 15-39 years, by diabetes type}
    \label{Denmarkcrude}
\end{figure}
\begin{figure}
    \centering
    \includegraphics[width=0.8\textwidth]{log/2_3.pdf}
    \caption{Crude incidence of diabetes in Finland among people aged 15-39 years, by diabetes type}
    \label{Finlandcrude}
\end{figure}
\begin{figure}
    \centering
    \includegraphics[width=0.8\textwidth]{log/2_4.pdf}
    \caption{Crude incidence of diabetes in Hungary among people aged 15-39 years, by diabetes type}
    \label{Hungarycrude}
\end{figure}
\begin{figure}
    \centering
    \includegraphics[width=0.8\textwidth]{log/2_5.pdf}
    \caption{Crude incidence of diabetes in Japan among people aged 15-39 years, by diabetes type}
    \label{Japancrude}
\end{figure}
\begin{figure}
    \centering
    \includegraphics[width=0.8\textwidth]{log/2_6.pdf}
    \caption{Crude incidence of diabetes in Scotland among people aged 15-39 years, by diabetes type}
    \label{Scotlandcrude}
\end{figure}
\begin{figure}
    \centering
    \includegraphics[width=0.8\textwidth]{log/2_7.pdf}
    \caption{Crude incidence of diabetes in South Korea among people aged 15-39 years, by diabetes type}
    \label{South Koreacrude}
\end{figure}
\begin{stlog}\input{log/2.log.tex}\end{stlog}
\begin{figure}
    \centering
    \includegraphics[width=0.8\textwidth]{log/3.pdf}
    \caption{Crude incidence of diabetes in Australia among people aged 15-39 years, by diabetes type and sex. Females = solid connecting lines; males = dashed connecting lines.}
    \label{Australiacrude}
\end{figure}
\begin{figure}
    \centering
    \includegraphics[width=0.8\textwidth]{log/3_1.pdf}
    \caption{Crude incidence of diabetes in Catalonia, Spain among people aged 15-39 years, by diabetes type and sex. Females = solid connecting lines; males = dashed connecting lines.}
    \label{Catalonia, Spaincrude}
\end{figure}
\begin{figure}
    \centering
    \includegraphics[width=0.8\textwidth]{log/3_2.pdf}
    \caption{Crude incidence of diabetes in Denmark among people aged 15-39 years, by diabetes type and sex. Females = solid connecting lines; males = dashed connecting lines.}
    \label{Denmarkcrude}
\end{figure}
\begin{figure}
    \centering
    \includegraphics[width=0.8\textwidth]{log/3_3.pdf}
    \caption{Crude incidence of diabetes in Finland among people aged 15-39 years, by diabetes type and sex. Females = solid connecting lines; males = dashed connecting lines.}
    \label{Finlandcrude}
\end{figure}
\begin{figure}
    \centering
    \includegraphics[width=0.8\textwidth]{log/3_4.pdf}
    \caption{Crude incidence of diabetes in Hungary among people aged 15-39 years, by diabetes type and sex. Females = solid connecting lines; males = dashed connecting lines.}
    \label{Hungarycrude}
\end{figure}
\begin{figure}
    \centering
    \includegraphics[width=0.8\textwidth]{log/3_5.pdf}
    \caption{Crude incidence of diabetes in Japan among people aged 15-39 years, by diabetes type and sex. Females = solid connecting lines; males = dashed connecting lines.}
    \label{Japancrude}
\end{figure}
\begin{figure}
    \centering
    \includegraphics[width=0.8\textwidth]{log/3_6.pdf}
    \caption{Crude incidence of diabetes in Scotland among people aged 15-39 years, by diabetes type and sex. Females = solid connecting lines; males = dashed connecting lines.}
    \label{Scotlandcrude}
\end{figure}
\begin{figure}
    \centering
    \includegraphics[width=0.8\textwidth]{log/3_7.pdf}
    \caption{Crude incidence of diabetes in South Korea among people aged 15-39 years, by diabetes type and sex. Females = solid connecting lines; males = dashed connecting lines.}
    \label{South Koreacrude}
\end{figure}
\begin{stlog}\input{log/3.log.tex}\end{stlog}
\color{black}
\clearpage
\section{Age and sex-specific rates}
\label{asrsec}

For the analyses, we are going to use Carstensen's Age-Period-Cohort model \cite{CarstensenSTATMED2007}
to estimate the age and sex-specific incidence of type 2 diabetes for each country. For this, 
we take the incidence and person-years in 5-year age groups, and fit a Poisson model with spline effects
of age, period (calendar time; measured from 2010 (i.e., 2010 is set to 0)), and cohort (calendar time minus age). 
This is done separately for each country and sex. Moreover, because of the different years covered by each dataset,
the knot locations are different for each country (and knot placement is as recommended by Harrell \cite{Harrell2001Springer} 
for period and cohort effects).
Then, we use this model to predict the incidence of type 2 diabetes
for specific ages. These results are presented in figures showing the age-specific
incidence of type 2 diabetes for each country (figures~\ref{Australia agespec} - ~\ref{South Korea agespec}).

\color{Blue4}
\begin{stlog}\input{log/4.log.tex}\end{stlog}
\begin{figure}
    \centering
    \includegraphics[width=0.8\textwidth]{log/5.pdf}
    \caption{Incidence of diabetes in Australia for people aged 15, 20, 25, 30, and 35 years, by diabetes type and sex}
    \label{Australia agespec}
\end{figure}
\begin{figure}
    \centering
    \includegraphics[width=0.8\textwidth]{log/5_1.pdf}
    \caption{Incidence of diabetes in Catalonia, Spain for people aged 15, 20, 25, 30, and 35 years, by diabetes type and sex}
    \label{Catalonia, Spain agespec}
\end{figure}
\begin{figure}
    \centering
    \includegraphics[width=0.8\textwidth]{log/5_2.pdf}
    \caption{Incidence of diabetes in Denmark for people aged 15, 20, 25, 30, and 35 years, by diabetes type and sex}
    \label{Denmark agespec}
\end{figure}
\begin{figure}
    \centering
    \includegraphics[width=0.8\textwidth]{log/5_3.pdf}
    \caption{Incidence of diabetes in Finland for people aged 15, 20, 25, 30, and 35 years, by diabetes type and sex}
    \label{Finland agespec}
\end{figure}
\begin{figure}
    \centering
    \includegraphics[width=0.8\textwidth]{log/5_4.pdf}
    \caption{Incidence of diabetes in Hungary for people aged 15, 20, 25, 30, and 35 years, by diabetes type and sex}
    \label{Hungary agespec}
\end{figure}
\begin{figure}
    \centering
    \includegraphics[width=0.8\textwidth]{log/5_5.pdf}
    \caption{Incidence of diabetes in Japan for people aged 15, 20, 25, 30, and 35 years, by diabetes type and sex}
    \label{Japan agespec}
\end{figure}
\begin{figure}
    \centering
    \includegraphics[width=0.8\textwidth]{log/5_6.pdf}
    \caption{Incidence of diabetes in Scotland for people aged 15, 20, 25, 30, and 35 years, by diabetes type and sex}
    \label{Scotland agespec}
\end{figure}
\begin{figure}
    \centering
    \includegraphics[width=0.8\textwidth]{log/5_7.pdf}
    \caption{Incidence of diabetes in South Korea for people aged 15, 20, 25, 30, and 35 years, by diabetes type and sex}
    \label{South Korea agespec}
\end{figure}
\begin{stlog}\input{log/5.log.tex}\end{stlog}
\color{black}

To make comparison between countries easier, we will plot all curves for age 25 on the same graph 
(and 20 and 30, to see if there is any difference depending on the age selected; figures ~\ref{agespec20} - ~\ref{agespec30}).

For these plots, we no longer use an ordinal colour scheme. We're using modified rainbow 
(because some of the rainbow colours are really hard to see).

\color{Blue4}
\begin{stlog}\input{log/6.log.tex}\end{stlog}
\begin{figure}
    \centering
    \includegraphics[width=0.8\textwidth]{log/7.pdf}
    \caption{Incidence of diabetes for people aged 20 years, by diabetes type and sex}
    \label{agespec20}
\end{figure}
\begin{figure}
    \centering
    \includegraphics[width=0.8\textwidth]{log/7_1.pdf}
    \caption{Incidence of diabetes for people aged 25 years, by diabetes type and sex}
    \label{agespec25}
\end{figure}
\begin{figure}
    \centering
    \includegraphics[width=0.8\textwidth]{log/7_2.pdf}
    \caption{Incidence of diabetes for people aged 30 years, by diabetes type and sex}
    \label{agespec30}
\end{figure}
\begin{stlog}\input{log/7.log.tex}\end{stlog}
\color{black}

\clearpage
\section{Age-standardized rates}


Additionally, we will age-standardise the incidence rates to the European population in 2010. 
This will be done using the same Age-Period-Cohort models described above. In this analysis, we will take
the predicted rates from these models (in single years) and use these in direct standardisation. However,
to do this, we first need to convert the European standard population (available only in 5-year age groups)
to 1-year age groups (using linear regression).

\color{Blue4}
\begin{figure}
    \centering
    \includegraphics[width=0.8\textwidth]{log/8.pdf}
    \caption{European standard population in 2010}
    \label{ESP2010N}
\end{figure}
\begin{figure}
    \centering
    \includegraphics[width=0.8\textwidth]{log/8_1.pdf}
    \caption{European standard population proportions in 2010}
    \label{ESP2010P}
\end{figure}
\begin{stlog}\input{log/8.log.tex}\end{stlog}
\color{black}

\clearpage

With that, we can calculate and plot the age-standardized rates.
Note: the method used to calculate the confidence intervals is the 
same as the Stata command \href{https://www.stata.com/manuals/rdstdize.pdf}{dstdize}, and it assumes that the person-years
are the same for each single age within the 5-year age group (which, if the 
populations we sample from are anything like the European population, is 
a safe assumption (figure~\ref{ESP2010P}, and is unlikely to affect the calculated
error even if included)).

\color{Blue4}
\begin{stlog}\input{log/9.log.tex}\end{stlog}
\begin{figure}
    \centering
    \includegraphics[width=0.8\textwidth]{log/10.pdf}
    \caption{Age-standardized incidence of diabetes for people aged 15-39 years, by diabetes type and sex}
    \label{STDfig}
\end{figure}
\begin{stlog}\input{log/10.log.tex}\end{stlog}
\color{black}

\clearpage
\section{Average Annual Percent Changes}

As a summary metric, we will also estimate the average annual percent change
in incidence - overall, and by sex. For this, we use a different model with a spline effect
of age, but only a (log-)linear effect of calendar time. This means we are assuming the effect of time
is constant throughout follow-up, which we already know is false for a few countries (e.g., Australia;
figure~\ref{agespec25}). \\

\color{Blue4}
\begin{stlog}\input{log/11.log.tex}\end{stlog}
\color{black}

\begin{table}[h!]
  \begin{center}
    \caption{Average annual percent change in the incidence of diabetes, by country, sex, and diabetes type. Adjusted for age.}
    \label{APCs}
     \fontsize{7pt}{9pt}\selectfont\pgfplotstabletypeset[
      multicolumn names,
      col sep=colon,
      header=false,
      string type,
	  display columns/0/.style={column name=Country,
		assign cell content/.code={
\pgfkeyssetvalue{/pgfplots/table/@cell content}
{\multirow{3}{*}{##1}}}},
	  display columns/1/.style={column name=Period,
		assign cell content/.code={
\pgfkeyssetvalue{/pgfplots/table/@cell content}
{\multirow{3}{*}{##1}}}},
      display columns/2/.style={column name=Sex, column type={l}, text indicator="},
      display columns/3/.style={column name=Type 1 diabetes, column type={r}, column type/.add={|}{}},
      display columns/4/.style={column name=Type 2 diabetes, column type={r}, column type/.add={|}{}},
      display columns/5/.style={column name=Uncertain diabetes type, column type={r}, column type/.add={|}{|}},
      every head row/.style={
        before row={\toprule
					},
        after row={\midrule}
            },
        every nth row={3}{before row=\midrule},
        every last row/.style={after row=\bottomrule},
    ]{APCs.csv}
  \end{center}
\end{table}

It's also worth looking at variation in the incidence rates by age, as some of the figures in section~\ref{asrsec}
suggested a greater increase in type 2 diabetes at younger ages. For this, we will use two models: 
the first includes the interaction between a spline effect of age and a log-linear effect of calendar time (plotted in the left
panels of the combined figures),
whereas the second includes a spline effect of age and the product of log-linear effects of age and calendar time (plotted on the right in the figures). 

\color{Blue4}
\begin{stlog}\input{log/12.log.tex}\end{stlog}
\begin{figure}
    \centering
    \includegraphics[width=0.8\textwidth]{log/13.pdf}
    \caption{Annual percent change in the incidence of diabetes in Australia by age, by diabetes type and sex. Values are predicted from a Poisson model with a spline effect of attained age, a log-linear effect of calendar time, and an interaction between age and calendar time. The left panels use a spline term for age in the interaction, the right panels use the product of age and calendar time in the interaction.}
    \label{Australia apcageg}
\end{figure}
\begin{figure}
    \centering
    \includegraphics[width=0.8\textwidth]{log/13_1.pdf}
    \caption{Annual percent change in the incidence of diabetes in Catalonia, Spain by age, by diabetes type and sex. Values are predicted from a Poisson model with a spline effect of attained age, a log-linear effect of calendar time, and an interaction between age and calendar time. The left panels use a spline term for age in the interaction, the right panels use the product of age and calendar time in the interaction.}
    \label{Catalonia, Spain apcageg}
\end{figure}
\begin{figure}
    \centering
    \includegraphics[width=0.8\textwidth]{log/13_2.pdf}
    \caption{Annual percent change in the incidence of diabetes in Denmark by age, by diabetes type and sex. Values are predicted from a Poisson model with a spline effect of attained age, a log-linear effect of calendar time, and an interaction between age and calendar time. The left panels use a spline term for age in the interaction, the right panels use the product of age and calendar time in the interaction.}
    \label{Denmark apcageg}
\end{figure}
\begin{figure}
    \centering
    \includegraphics[width=0.8\textwidth]{log/13_3.pdf}
    \caption{Annual percent change in the incidence of diabetes in Finland by age, by diabetes type and sex. Values are predicted from a Poisson model with a spline effect of attained age, a log-linear effect of calendar time, and an interaction between age and calendar time. The left panels use a spline term for age in the interaction, the right panels use the product of age and calendar time in the interaction.}
    \label{Finland apcageg}
\end{figure}
\begin{figure}
    \centering
    \includegraphics[width=0.8\textwidth]{log/13_4.pdf}
    \caption{Annual percent change in the incidence of diabetes in Hungary by age, by diabetes type and sex. Values are predicted from a Poisson model with a spline effect of attained age, a log-linear effect of calendar time, and an interaction between age and calendar time. The left panels use a spline term for age in the interaction, the right panels use the product of age and calendar time in the interaction.}
    \label{Hungary apcageg}
\end{figure}
\begin{figure}
    \centering
    \includegraphics[width=0.8\textwidth]{log/13_5.pdf}
    \caption{Annual percent change in the incidence of diabetes in Japan by age, by diabetes type and sex. Values are predicted from a Poisson model with a spline effect of attained age, a log-linear effect of calendar time, and an interaction between age and calendar time. The left panels use a spline term for age in the interaction, the right panels use the product of age and calendar time in the interaction.}
    \label{Japan apcageg}
\end{figure}
\begin{figure}
    \centering
    \includegraphics[width=0.8\textwidth]{log/13_6.pdf}
    \caption{Annual percent change in the incidence of diabetes in Scotland by age, by diabetes type and sex. Values are predicted from a Poisson model with a spline effect of attained age, a log-linear effect of calendar time, and an interaction between age and calendar time. The left panels use a spline term for age in the interaction, the right panels use the product of age and calendar time in the interaction.}
    \label{Scotland apcageg}
\end{figure}
\begin{figure}
    \centering
    \includegraphics[width=0.8\textwidth]{log/13_7.pdf}
    \caption{Annual percent change in the incidence of diabetes in South Korea by age, by diabetes type and sex. Values are predicted from a Poisson model with a spline effect of attained age, a log-linear effect of calendar time, and an interaction between age and calendar time. The left panels use a spline term for age in the interaction, the right panels use the product of age and calendar time in the interaction.}
    \label{South Korea apcageg}
\end{figure}
\begin{stlog}\input{log/13.log.tex}\end{stlog}
\color{black}

\clearpage
\bibliography{/Users/jed/Documents/Library.bib}
\end{document}
